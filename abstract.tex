
\begin{abstract}

The Vera C. Rubin Observatory (Rubin Observatory) was designed and constructed to conduct a 10-year wide-area, deep, multi-band optical imaging survey of the night sky visible from Chile.
This Legacy Survey of Space and Time (LSST) will catalog billions of individual stars and galaxies and detect millions of time-domain events each night, enabling a broad range of astrophysics research including studies of dark matter and dark energy, the transient and variable universe, the structure of the Milky Way, and an inventory of the Solar System.
Here we report a first scientific characterization of the as-built Rubin Observatory based on commissioning observations taken in May 2022 for two LSST Camera pointings centered on an LSST Deep Drilling Field and a dense stellar field, respectively, each with more than $800 \times 30$-second visits across the $ugrizy$ bands. 
%with the primary objective of evaluating operational readiness for the LSST.
The delivered image quality, achieved signal-to-noise, astrometric and photometric calibration, and quality of object detection and difference image analysis are found to be within the design specifications of the observatory.
%As the Commissioning Execution Plan (LSE-390) says, "The project team shall
%deliver all reports documenting the as-built hardware and software including:
%drawings, source code, modifications, compliance exceptions, and recommendations
%for improvement." As a first step towards the delivery of documents that will describe the system at the
%end of construction, we are assembling teams for producing of the order 40 papers
%that eventually will be submitted to relevant professional journals. The immediate goal is to accomplish
%all the writing that can be done without data analysis before the data
%taking begins, and the team becomes much more busy and stressed.
%This document provides the template for these papers.

\end{abstract}

