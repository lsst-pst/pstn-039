\section{Introduction}

\note{Here is a note}

Scope question: given that there are dedicated papers planned on various aspects of system performance and the data management system, it seems like this paper might be focused on giving users a sense of what the LSST data looks like and science validation topics. 

Question of whether this paper should focus on the ``static sky'', or also include the annual re-processing of individual visits to better characterize time-domain phenomena.
If the spirit of this paper is more validation and a science-user audience, some verification plots might go into other papers.

Examples:

\begin{enumerate}
\item DES DR1 arXiv:1801.03181
\item DES Y1 Gold arXiv:1708.01531
\item HSC DR1 arXiv:1702.08449
\item HSC DR2 arXiv:1905.12221
\item Gaia papers
\end{enumerate}

\section{Commissioning On-sky Observations}

I think we want to have figures of the distribution of single-visit image quality (e.g., FWHM in each band), sky brightness in each band (e.g., FWHM in each band), and system throughput in each band (e.g., single-visit depth corresponding to SNR = 5 in each band). I am guessing these would go in ``Performance of Delivered LSST System'' (PSTN-032).

Information on survey speed goes in ``Performance Verification of the LSST Survey Scheduler'' (PSTN-043).

\subsection{Deep Field Observations}

Let's hope we have at least one deep field.

\subsection{Wide Area Observations}

Let's hope we get multiband imaging of several hundred square degrees.

\section{Data Release Processing}

Focus on what was done specifically for the commissioning ok-sky observations.
My understanding is that algorithmic details will be discussed in ``LSST Data Release Processing'' (PSTN-020).

\section{Data Release Performance}

\subsection{Integrated Astrometric Performance}

Astrometric residuals as a function of, e.g., focal plane position in another paper?

Focusing on the coadd, though perhaps it makes sense to discuss single-visit astrometry as well.
Parallax and proper motion may be on longer timescale than that of commissioning observations.

\subsection{Photometry}

Photometric residuals as function of, e.g., focal plane position in another paper?

Uniformity and colors in coadd.
I figure the comparison to AB scale will be discussed elsewhere.

Stellar locus in color space

Galaxies in color space.

\subsection{Depth}

Focusing on the coadd.

\subsubsection{Flux distribution}

\subsubsection{Flux limit at fixed signal-to-noise}

\subsubsection{Depth from image properties}

\subsubsection{Object detection completeness}

%\subsection{Object Detection}

\subsection{Flagged Objects}

Spurious objects, artifacts, etc.

\subsection{Object Classification}

Star-galaxy separation

Density maps of stars and galaxies

\subsection{Deblending}

Dense stellar fields? Galaxy clusters? Deep coadd. Results from source injection.
Not exactly sure where this should go, and how much detail to go into.

\subsection{Image Quality and Object }

Ellipticity correlations and weak lensing null tests here. 
I'm thinking specifically about some demonstrations that the recovered shear is reasonable.

\subsection{Photometric Redshifts}

Potential science validation topic. Not clear that we will get this far during commissioning. Still, there are probably interesting comparisons to be done to assess galaxy colors.

\subsection{Known Issues}

Lets hope this section is short...

\section{Release Products}

\subsection{Images}

\subsection{Catalogs}

