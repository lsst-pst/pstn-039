\documentclass[modern]{aastex62}

% lsstdoc documentation: https://lsst-texmf.lsst.io/lsstdoc.html
\input{meta}

% Package imports go here.
\usepackage{tcolorbox}

% Local commands go here.
\newcommand{\note}[1]{
   \begin{tcolorbox}[colback=red!5!white, colframe=red!75!black]
      #1
   \end{tcolorbox}
}

\newcommand{\docRef}{PSTN-039}
\newcommand{\docUpstreamLocation}{\url{https://github.com/lsst-pst/pstn-039}}


\begin{document}
\input{authors}
\date{\today}
\title{ Science Validation of LSST Data Release Processing}
\hypersetup{pdftitle={\@title}, pdfauthor={\@author}, pdfkeywords={\@keywords}}


\begin{abstract}
 
As the Commissioning Execution Plan (LSE-390) says, "The project team shall
deliver all reports documenting the as-built hardware and software including:
drawings, source code, modifications, compliance exceptions, and recommendations
for improvement." As a first step towards the delivery of documents that will describe the system at the
end of construction, we are assembling teams for producing of the order 40 papers
that eventually will be submitted to relevant professional journals. The immediate goal is to accomplish
all the writing that can be done without data analysis before the data
taking begins, and the team becomes much more busy and stressed.

This document provides the template for these papers.
\end{abstract}



\section{Introduction}

Scope question: given that there are dedicated papers planned on various aspects of system performance and the data management system, it seems like this paper might be focused on giving users a sense of what the LSST data looks like and science validation topics. 

Question of whether this paper should focus on the ``static sky'', or also include the annual re-processing of individual visits to better characterize time-domain phenomena.
If the spirit of this paper is more validation and a science-user audience, some verification plots might go into other papers.

Examples:

\begin{enumerate}
\item DES DR1 arXiv:1801.03181
\item DES Y1 Gold arXiv:1708.01531
\item HSC DR1 arXiv:1702.08449
\item HSC DR2 arXiv:1905.12221
\item Gaia papers
\end{enumerate}

\section{Commissioning On-sky Observations}

I think we want to have figures of the distribution of single-visit image quality (e.g., FWHM in each band), sky brightness in each band (e.g., FWHM in each band), and system throughput in each band (e.g., single-visit depth corresponding to SNR = 5 in each band). I am guessing these would go in ``Performance of Delivered LSST System'' (PSTN-032).

Information on survey speed goes in ``Performance Verification of the LSST Survey Scheduler'' (PSTN-043).

\subsection{Deep Field Observations}

Let's hope we have at least one deep field.

\subsection{Wide Area Observations}

Let's hope we get multiband imaging of several hundred square degrees.

\section{Data Release Processing}

Focus on what was done specifically for the commissioning ok-sky observations.
My understanding is that algorithmic details will be discussed in ``LSST Data Release Processing'' (PSTN-020).

\section{Data Release Performance}

\subsection{Integrated Astrometric Performance}

Astrometric residuals as a function of, e.g., focal plane position in another paper?

Focusing on the coadd, though perhaps it makes sense to discuss single-visit astrometry as well.
Parallax and proper motion may be on longer timescale than that of commissioning observations.

\subsection{Photometry}

Photometric residuals as function of, e.g., focal plane position in another paper?

Uniformity and colors in coadd.
I figure the comparison to AB scale will be discussed elsewhere.

Stellar locus in color space

Galaxies in color space.

\subsection{Depth}

Focusing on the coadd.

\subsubsection{Flux distribution}

\subsubsection{Flux limit at fixed signal-to-noise}

\subsubsection{Depth from image properties}

\subsubsection{Object detection completeness}

%\subsection{Object Detection}

\subsection{Flagged Objects}

Spurious objects, artifacts, etc.

\subsection{Object Classification}

Star-galaxy separation

Density maps of stars and galaxies

\subsection{Deblending}

Dense stellar fields? Galaxy clusters? Deep coadd. Results from source injection.
Not exactly sure where this should go, and how much detail to go into.

\subsection{Image Quality and Object }

Ellipticity correlations and weak lensing null tests here. 
I'm thinking specifically about some demonstrations that the recovered shear is reasonable.

\subsection{Photometric Redshifts}

Potential science validation topic. Not clear that we will get this far during commissioning. Still, there are probably interesting comparisons to be done to assess galaxy colors.

\subsection{Known Issues}

Lets hope this section is short...

\section{Release Products}

\subsection{Images}

\subsection{Catalogs}


%\section{Introduction}

{\bf Eventually, please replace all of the remaining text with your paper text.}
\vskip 0.4in


The LSST Construction Project team needs to document the as-built hardware and software
(see LSE-79 and LSE-390 for details). Although this activity will likely continue well into the operations phase, the majority 
of anticipated documents will be necessary to enable efficient and robust early science with the LSST 
facility and thus must be available, at least in a draft form, by the first data release. 

As a first step, we are now assembling teams that will be in charge of delivering these documents.
An initial paper list collated by subsystem leaders includes about 40 papers that will be submitted 
to relevant professional journals. Therefore, this deliverable represents a major undertaking
and we need to start early. In addition, the commissioning period will be shorter than anticipated 
due to various delays in construction  and thus the time to complete these papers will be
shorter, too. Although most of these papers cannot be finished before the end of construction 
because they will require analysis of LSST commissioning data, we can significantly mitigate
the risk that they will never be finished by starting early. The early start will also help 
mitigate another source of stress for the team during the busy commissioning phase. 

 
\section{Initial Plan} 

The subsystem leaders have assembled an initial list of papers, listed in Appendix. 
It is likely that this list will evolve with time. 
Each paper has an editor assigned to it. Each editor is meant to be a team leader
who will be initially responsible for the completion of the assigned paper (or perhaps 
until somone else from the team assumes this leadership role). The editor is not 
necessarily the team member who will do most of the required work, or who will 
eventually become the first author. Both issues will be handled by on an individual team basis. 

\subsection{The timeline} 

We would like to have all the sections that do not depend on commissioning data 
written and reviewed by February 2021. If we accomplish this goal, we will both
have easier time completing these papers, and the team will be less stressed during
the commissioning phase. 

Our initial timeline is as follows (the further into the future, the less certain it is): 
\begin{enumerate} 
\item 
Subsystem leads assemble the initial list of papers (DONE)
\item
Setup latex templates and email exploders (lsst-constrpapers) (DONE)
\item
Schedule the first telecon to discuss task, overall plan and timeline (Oct 2019).
\item 
Delivery of paper outlines and the second telecon (Jan 2020). Each paper
outline should at least contain the list of all sections, their lead authors, and
a few sentences about the section scope. Overachievers can add a list of figures etc.
for extra credit. 
\item
First rough draft of sections that can be written without having the LSST commissioning
data and the third telecon (June 2020). These drafts should at least include subsection
structure, lists of planned tables, figures, rough text,  and identification of any impediments 
to make the Oct. deadline for drafts ready for review (so that we can replan if need be). 
\item 
Sections that can be written without having the data ready for an internal project review and
the fourth telecon (Nov 2020). 
\item
Reviews available and the fifth telecon (Feb 2021)
\item
Implementation of the reviewers' comments (from Feb 2021 until first light) 
\item 
Final drafts, including sections that depend on LSST data, available for
review and the sixth telecon (Aug 2022)
\item
Implementation of the reviewers' comments (from Aug 2021 until the start of operations, 
planned for Oct 3, 2022).  Proceeding with submissions, details TBD...
\end{enumerate}




\section{Some technicalities: author list and standard LSST references} 

Thank you Tim Jenness and Wil O'Mullane for helping with templates! 

\subsection{The LSST LaTeX Classes}

Please see the installation instructions\footnote{\url{https://lsst-texmf.lsst.io/install.html}} 
for lsst-texmf. Once you have it installed, you should be able to compile your paper
using make. 

\subsection{How to handle author list?} 

Authors come from the authors.yaml file --  find the author ids in the lsst-texmf/etc/authordb.yaml - use db2authors to get the authors and institutes from the db. 

{\bf XXX Wil, the above is unclear: need more detail about how to use db2authors,
what is its output and what to do with it...} 


\subsection{How to handle LSST standard references?} 

The papers should cite standard LSST references\footnote{See \url{https://github.com/lsst-pst/LSSTreferences}}, 
where appropriate. For the usage, please see below.  These examples all use the ADS handle, unless they are 
project docs then the use the project handle like LSE-17.

All are on the lsst-texmf which you can get from \url{http://lsst-texmf.lsst.io}


\subsubsection{LSST System and Science}

The LSST system (brief overview of telescope, camera and data management subsystems),
science drivers and science forecasts are described in:

\begin{itemize}
\item LSST Science Requirements Document: \cite{LPM-17}.
\item LSST overview paper: \cite{2008arXiv0805.2366I}.
\item LSST Science Book: \cite{abell2009lsst}.
\end{itemize}
%------------------------------------------------------------------------------


\subsubsection{Simulations}

The LSST simulations are described in a series of papers. Use of the LSST simulations should cite the LSST simulations overview paper \cite{2014SPIE.9150E..14C} and the specific simulation tools used:

\begin{itemize}
\item LSST Catalogs (CatSim): \cite{2014SPIE.9150E..14C}
\item Feature-Based Scheduler: \cite{2018arXiv181004815N}
\item Operations Simulator (OpSim): Scheduler \cite{2016SPIE.9910E..13D}, SOCS \cite{2016SPIE.9911E..25R}
\item Metrics Analysis Framework (MAF): \cite{2014SPIE.9149E..0BJ}
\item Image simulations (Phosim): \cite{2015ApJS..218...14P}
\item Sky brightness model: \cite{2016SPIE.9910E..1AY}
\item LSST Performance for NEO (or moving object) discovery: \cite{2018Icar..303..181J}
\end{itemize}
%------------------------------------------------------------------------------


\subsubsection{Data Management}

LSST data management system and the data products are described in:

\begin{itemize}
  \item The LSST Data Management System: \cite{2015arXiv151207914J}
  \item Data Products Definition Document: \cite{LSE-163}
\end{itemize}
 %------------------------------------------------------------------------------


\subsubsection{Camera}

\begin{itemize}
   \item Design and development of the LSST camera: \cite{2010SPIE.7735E..0JK}
\end{itemize}
%------------------------------------------------------------------------------


\subsubsection{Telescope and Site}

\begin{itemize}
   \item Telescope and site overview and status in 2014:  \cite{2014SPIE.9145E..1AG}
\end{itemize}
%------------------------------------------------------------------------------

\subsubsection{System Engineering}

\begin{itemize}
   \item LSST systems engineering: \cite{2014SPIE.9150E..0MC}
   \item System verification and validation: \cite{2014SPIE.9150E..0NS}
\end{itemize}
%


 % Do not include the default template

\appendix
% Remove this when you strart your paper

{\bf Initial paper list added here for reference.}

``Editor'' is a responsible team leader but not necessarily the person who will do most of
the required work, or who will eventually become the first author. Both issues will be
handled by individual teams.

\begin{verbatim}

domain: Telescope & Site
editor: Jeff Barr
title: Overview of the LSST Telescope

domain: Telescope & Site
editor: Sandrine Thomas
title: Performance of the LSST Telescope

domain: Telescope & Site
editor: Lynne Jones
title: The LSST Scheduler Overview and Performance

domain: Telescope & Site
editor: Bo Xin
title: Performance of the LSST Active Optics System

domain: Telescope & Site
editor: Tiago Ribeiro
title: LSST Observing System Software Architecture

domain: Camera
editor: Justin Wolfe
title: LSST Camera Optics

domain: Camera
editor: Chris Stubbs
title: LSST Camera Rafts

domain: Camera
editor: Steve Ritz
title: LSST Camera Cryostat

domain: Camera
editor: Ralph Schindler
title: LSST Camera Refrigeration

domain: Camera
editor: Steve Ritz
title: LSST Camera Body and Mechanisms

domain: Camera
editor: Mark Huffer and Tony Johnson
title: LSST Camera Control System and DAQ

domain: Camera
editor: Tim Bond and Aaron Rodman
title: LSST Camera Integration and Tests

domain: Data Management
editor: Leanne Guy
title: Overview of LSST Data Management

domain: Data Management
editor: Michelle Butler
title: LSST Data Facility

domain: Data Management
editor: Tim Jenness
title: LSST Data Management Software System

domain: Data Management
editor: Jim Bosch
title: LSST Data Release Processing

domain: Data Management
editor: Eric Bellm
title: LSST Prompt Data Products

domain: Data Management
editor: Gregory Dubois-Felsmann
title: LSST Science Platform

domain: Data Management
editor: Simon Krughoff
title: LSST Data Management Quality Assurance and Reliability Engineering

domain: Data Management
editor: Leanne Guy (with likely delegation to new DM V&V Scientist)
title: LSST Data Management System Verification and Validation

domain: Data Management
editor: Mario Juric
title: LSST Moving Object Processing

domain: Data Management
editor: Robert Lupton
title: LSST Calibration Strategy and Pipelines

domain: Calibration
editor: Patrick Ingraham
title:  Performance of the LSST Calibration Systems

domain: Calibration
editor: Patrick Ingraham
title: Atmospheric Properties with the LSST Auxiliary Telescope

domain: EPO
editor: Amanda Bauer
title: Overview of LSST Education and Public Outreach

domain: EPO
editor: Ardis Herrold
title: LSST Formal Education Program

domain: EPO
editor: Amanda Bauer
title: LSST EPO: The User Feedback

domain: Commissioning
editor: Chuck Claver
title: LSST Observatory System Operations Readiness Report

domain: Commissioning
editor: Bo Xin
title: Performance of Delivered LSST System

domain: Commissioning
editor: Chuck Claver
title: Active Optics Performance with LSST Commissiong Camera

domain: Commissioning
editor: Chuck Claver
title: LSST Active Optics Performance with the LSST Science Camera

domain: Commissioning
editor: Brian Stalder
title: Integration, Test and Commissioning Results from LSST Commissiong Camera

domain: Commissioning
editor: Kevin Reil
title: LSST Camera Instrumental Signature Characterization, Calibration and Removal

domain: Commissioning
editor: Patrick Hascal
title: Installation and Performance of the LSST Camera Refrigeration System

domain: Commissioning
editor: Andy Connolly
title: Science Validation of LSST Alert Processing

domain: Commissioning
editor: Keith Bechtol
title: Science Validation of LSST Data Release Processing

domain: Commissioning
editor: Michael Reuter
title: Tracking of LSST System Performance with Continuous Integration Methods

domain: Commissioning
editor: Chuck Claver
title: The LSST Science Platform as a Commissioning Tool

domain: Commissioning
editor: Chuck Claver
title: Commissioning Science Data Quality Analysis Tools, Methods and Procedures

domain: Commissioning
editor: Lynne Jones
title: Performance Verification of the LSST Survey Scheduler


\end{verbatim}

% Include all the relevant bib files.
% https://lsst-texmf.lsst.io/lsstdoc.html#bibliographies
\section{References} \label{sec:bib}
\bibliographystyle{yahapj}
\bibliography{local,lsst,lsst-dm,refs_ads,refs,books}

% Make sure lsst-texmf/bin/generateAcronyms.py is in your path
\section{Acronyms} \label{sec:acronyms}
\input{acronyms.tex}

\end{document}
